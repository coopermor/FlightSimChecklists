\documentclass{article}
\usepackage[margin=0.1in]{geometry}
\usepackage[explicit]{titlesec}
\usepackage{xcolor}
\usepackage{multicol}

\definecolor{Red}{rgb}{1,0,0}
\definecolor{Black}{rgb}{0,0,0}
\titleformat{\section}
{\normalfont\normalsize\bfseries}{}{0em}{\colorbox{black}{\parbox{\dimexpr\linewidth-2\fboxsep\relax}{\textcolor{white}{\thesection\quad#1}}}\vspace{-1em}}
\titleformat{name=\section,numberless}
{\normalfont\normalsize\bfseries}{}{0em}{\colorbox{black}{\parbox{\dimexpr\linewidth-2\fboxsep\relax}{\textcolor{white}{#1}}}\vspace{-1em}}

\usepackage{titlesec}
\titlespacing*{\section}{0pt}{0.1\baselineskip}{\baselineskip}

\renewcommand{\familydefault}{\sfdefault}
\begin{document}
\centering
{\fontsize{25}{70}\selectfont Cessna 152}
\break
{\fontsize{20.74}{70}\selectfont Quick Reference Checklist}
\break
\color{Red}
{\fontsize{10}{70}\selectfont For simulation use only, not for real world flight}

\color{Black}
\begin{multicols*}{2}
\section*{PREFLIGHT}
Ignition Switch\dotfill OFF\\
Master Switch\dotfill ON\\
\centering \textbf{WARNING:}
When turning on the master switch, using an external power source, or pulling the propeller through by hand, treat the propeller as if the ignition switch were on. Do not stand, nor allow anyone else to stand within the arc of the propeller, since a loose or broken wire, or a component malfunction, could cause the propeller to rotate.\\
Fuel Quantity Indications\dotfill CHECK QUANTITY\\
Master Switch\dotfill OFF\\
Fuel Shutoff Valve\dotfill ON\\
Empennage Control Surfaces\dotfill CHECK freedom of movement and security\\
Ailerons\dotfill CHECK freedom of movement and security\\
\section*{BEFORE STARTING ENGINE}
Preflight Inspection\dotfill COMPLETE\\
Fuel Shutoff Valve\dotfill ON\\
Radios, Electrical Equipment\dotfill OFF\\
Brakes\dotfill TEST and SET\\
Circuit Breakers\dotfill CHECK IN\\
\section*{STARTING ENGINE}
Mixture\dotfill RICH\\
Carburetor Heat\dotfill COLD\\
Master Switch\dotfill ON\\
Prime\dotfill AS REQUIRED (up to 3 strokes - none if engine is warm)\\
Throttle\dotfill OPEN 1/2 INCH (CLOSED if engine is warm)\\
Propeller Area\dotfill CLEAR\\
Master Switch\dotfill ON\\
Ignition Switch\dotfill START\\
Throttle\dotfill ADJUST for 1000 RPM or less
Oil Pressure\dotfill CHECK
Flashing Beacon and Navigation Lights\dotfill ON as required\\
Radios\dotfill ON
\section*{BEFORE TAKEOFF}
Parking Brake\dotfill SET\\
Flight Controls\dotfill FREE and CORRECT\\
Flight Instruments\dotfill SET\\
Fuel Shutoff Valve\dotfill ON\\
Mixture\dotfill RICH (below 3000 feet)\\
Elevator Trim\dotfill TAKEOFF\\
Throttle\dotfill 1700 RPM\\
\hspace*{6mm}Magnetos\dotfill CHECK (RPM drop should not exceed 125 RPM on either magneto or 50 RPM differential between magnetos).\\
\hspace*{6mm}Carburetor Heat\dotfill CHECK (for RPM drop)\\
\hspace*{6mm}Engine Instruments and Ammeter\dotfill CHECK\\
\hspace*{6mm}Suction Gage\dotfill CHECK\\
\hspace*{6mm}Throttle\dotfill 1000 RPM OR LESS\\
Radios\dotfill SET\\
Strobe Lights\dotfill AS DESIRED\\
Brakes\dotfill Release
\section*{NORMAL TAKEOFF}
Wing Flaps\dotfill 0° - 10°\\
Carburetor Heat\dotfill COLD\\
Throttle\dotfill FULL OPEN\\
Elevator Control\dotfill LIFT NOSE WHEEL at 50 KIAS\\
Climb Speed\dotfill 65-75 KIAS
\vfill\null
\section*{SHORT FIELD TAKEOFF}
Wing Flaps\dotfill 10°\\
Carburetor Heat\dotfill COLD\\
Brakes\dotfill APPLY\\
Throttle\dotfill FULL OPEN\\
Mixture\dotfill RICH (above 3000 feet, LEAN to obtain maximum RPM)\\
Brakes\dotfill RELEASE\\
Elevator Control\dotfill SLIGHTLY TAIL LOW\\
Climb Speed\dotfill 54 KIAS (until all obstacles are cleared)
Wing Flaps\dotfill RETRACT slowly after reaching 60 KIAS
\section*{CLIMB}
Airspeed\dotfill 70-8 KIAS\\
Throttle\dotfill FULL OPEN\\
Mixture\dotfill RICH below 3000 feet, LEAN for maximum RPM above 3000 feet\\
\section*{CRUISE}
Power\dotfill 1900-2550 RPM (no more than 75\%)\\
Elevator Trim\dotfill ADJUST\\
Mixture\dotfill LEAN
\section*{DESCENT}
Mixture\dotfill ADJUST for smooth operation (full rich for idle power)\\
Power\dotfill AS DESIRED\\
Carburetor Heat\dotfill FULL HEAT AS REQUIRED
\section*{PRE-LANDING}
Mixture\dotfill RICH\\
Carburetor Heat\dotfill ON (apply full heat before closing throttle)\\
\section*{NORMAL LANDING}
Airspeed\dotfill 60-70 KIAS (flaps UP)\\
Wing Flaps\dotfill AS DESIRED (below 85 KIAS)\\
Airspeed\dotfill 55-65 KIAS (flaps DOWN)\\
Touchdown\dotfill MAIN WHEELS FIRST\\
Landing Roll\dotfill LOWER NOSE WHEEL GENTLY\\
Braking\dotfill MINIMUM REQUIRED
\section*{SHORT FIELD LANDING}
Airspeed\dotfill 60-70 KIAS (flaps UP)\\
Wing Flaps\dotfill 30° (below 85 KIAS)\\
Airspeed\dotfill MAINTAIN 45 KIAS\\
Power\dotfill REDUCE to idle as obstacle is cleared
Touchdown\dotfill MAIN WHEELS FIRST\\
Braking\dotfill APPLY HEAVILY\\
Wing Flaps\dotfill RETRACT
\section*{BALKED LANDING}
Throttle\dotfill FULL OPEN\\
Carburetor Heat\dotfill COLD\\
Wing Flaps\dotfill RETRACT to 20°\\
Airspeed\dotfill 55 KIAS\\
Wing Flaps\dotfill RETRACT (slowly)
\section*{AFTER LANDING}
Wing Flaps\dotfill UP\\
Carburetor Heat\dotfill COLD\\
\section*{SECURING AIRPLANE}
Parking Brake\dotfill SET\\
Radios, Electrical Equipment\dotfill OFF\\
Mixture\dotfill IDLE CUT-OFF (pull full out)\\
Ignition Switch\dotfill OFF\\
Master Switch\dotfill OFF
\end{multicols*}
\newpage
\centering
{\fontsize{20.74}{70}\selectfont Emergency Procedures Checklist}
\break
\color{Red}
{\fontsize{10}{70}\selectfont For simulation use only, not for real world flight}

\color{Black}
\begin{multicols*}{2}
\section*{ENGINE FAILURE DURING TAKEOFF RUN}
Throttle\dotfill IDLE\\
Brakes\dotfill APPLY\\
Wing Flaps\dotfill RETRACT\\
Mixture\dotfill IDLE CUT-OFF\\
Ignition Switch\dotfill OFF\\
Master Switch\dotfill OFF
\section*{ENGINE FAILURE IMMEDIATELY AFTER TAKEOFF}
Airspeed\dotfill 60 KIAS\\
Mixture\dotfill IDLE CUT-OFF\\
Ignition Switch\dotfill OFF\\
Wing Flaps\dotfill AS REQUIRED\\
Master Switch\dotfill OFF
\section*{ENGINE FAILURE DURING FLIGHT}
Airspeed\dotfill 60 KIAS\\
Carburetor Heat\dotfill ON\\
Primer\dotfill IN and LOCKED\\
Fuel Shutoff Valve\dotfill ON\\
Mixture\dotfill RICH\\
Ignition Switch\dotfill BOTH (or START if propeller is stopped)\\
\section*{EMERGENCY LANDING WITHOUT ENGINE POWER}
Airspeed\dotfill 65 KIAS (flaps UP), 60 KIAS (flaps DOWN)\\
Mixture\dotfill IDLE CUT-OFF\\
Fuel Shutoff Valve\dotfill OFF\\
Ignition Switch\dotfill OFF\\
Wing Flaps\dotfill AS REQUIRED (30° recommended)\\
Master Switch\dotfill OFF\\
Touchdown\dotfill SLIGHTLY TAIL LOW\\
Brakes\dotfill APPLY HEAVILY
\section*{PRECAUTIONARY LANDING WITH ENGINE POWER}
Airspeed\dotfill 60 KIAS\\
Wing Flaps\dotfill 20°\\
Selected Field\dotfill FLY OVER, noting terrain and obstructions, then retract flaps upon reaching a safe altitude and airspeed\\
Radio and Electrical Switches\dotfill OFF\\
Wing Flaps\dotfill 30° (on final approach)\\
Airspeed\dotfill 55 KIAS\\
Master Switch\dotfill OFF\\
Touchdown\dotfill SLIGHTLY TAIL LOW\\
Ignition Switch\dotfill OFF\\
Brakes\dotfill  APPLY HEAVILY
\section*{DITCHING}
Radio\dotfill TRANSMIT MAYDAY on 121.5 MHz, giving location and intentions and SQUAWK 7700 if transponder is installed\\
Heavy Objects (in baggage area)\dotfill SECURE OR JETTISON\\
Approach\dotfill High Winds, Heavy Seas\dotfill INTO THE WIND\\
\hspace*{6mm}Light Winds, Heavy Swells\dotfill PARALLEL TO SWELLS\\
Wing Flaps\dotfill 30°\\
Power\dotfill ESTABLISH 300 FT/MIN DESCENT AT 55 KIAS.\\
Touchdown\dotfill LEVEL ATTITUDE AT ESTABLISHED RATE OF DESCENT
\section*{ENGINE FIRE IN FLIGHT}
Mixture\dotfill IDLE CUT-OFF\\
Fuel Selector Valve\dotfill OFF\\
Master Switch\dotfill OFF\\
Cabin Heat and Air\dotfill OFF (except wing root vents)\\
Airspeed\dotfill 85 KIAS (If fire is not extinguished, increase glide speed to find an airspeed which iwill provide an incombustible mixture)\\
Forced Landing\dotfill EXECUTE (as described in Emergency Landing Without Engine Power)
\section*{FIRE DURING START ON GROUND}
Cranking\dotfill CONTINUE, to get a start which would suck the flames and accumulated fuel through the carburetor and into the engine\\
\textbf{If engine starts:}\\
Power\dotfill 1700 RPM for a minutes\\
Engine\dotfill SHUTDOWN and inspect for damage\\
\textbf{If engine fails to start:}\\
Cranking\dotfill CONTINUE in an effort to obtain a start\\
Engine\dotfill SECURE\\
\hspace*{6mm}Master Switch\dotfill OFF\\
\hspace*{6mm}Ignition Switch\dotfill OFF\\
\hspace*{6mm}Fuel Selector Valve\dotfill OFF\\
Fire\dotfill EXTINGUISH using fire extinguisher\\
Fire Damage\dotfill INSPECT, repair damage or replace damaged components or wiring before conducting another flight
\section*{ELECTRICAL FIRE IN FLIGHT}
Master Switch\dotfill OFF\\
All Other Switches (except ignition switch)\dotfill OFF\\
Vents/Cabin Air/Heat\dotfill CLOSED\\
Fire Extinguisher\dotfill ACTIVATE (if available)\\
\centering \textbf{WARNING:}
After discharging an extinguisher within a closed cabin, ventilate the cabin.\\
If fire appears out and electrical power is necessary for continuance of flight:\\
Master Switch\dotfill ON\\
Circuit Breakers\dotfill CHECK for faulty circuit, do not reset\\
Radio Switches\dotfill OFF\\
Avionics Power Switch\dotfill ON\\
Radio/Electrical Switches\dotfill ON one at a time, with delay after each until short circuit is localized\\
Vents/Cabin Air/Heat\dotfill OPEN when it is ascertained that fire is completely extinguished
\section*{CABIN FIRE}
Master Switch\dotfill OFF\\
Vents/Cabin Air/Heat\dotfill CLOSED (to avoid drafts)\\
Fire Extinguisher\dotfill ACTIVATE (if available)\\
\centering \textbf{WARNING:}
After discharging an extinguisher within a closed cabin, ventilate the cabin.\\
Land the airplane as soon as possible to inspect for damage
\section*{WING FIRE}
Navigation Light Switch\dotfill OFF\\
Strobe Light Switch\dotfill OFF\\
Pitot Heat Switch\dotfill OFF\\
\centering \textbf{NOTE}:
Perform a sideslip to keep the flames away from the fuel tank and cabin, and land as soon as possible, with flaps retracted
\section*{STATIC SOURCE BLOCKAGE}
Alternate Static Source Valve\dotfill PULL ON
\section*{LANDING WITH A FLAT MAIN TIRE}
Wing Flaps\dotfill AS DESIRED\\
Approach\dotfill NORMAL\\
Touchdown\dotfill GOOD TIRE FIRST, hold airplane off flat tire as long as possible with aileron control
\section*{AMMETER SHOWS EXCESSIVE RATE OF CHARGE}
Alternator\dotfill OFF\\
Alternator Circuit Breaker\dotfill PULL\\
Nonessential Electrical Equipment\dotfill OFF\\
Flight\dotfill TERMINATE as soon as practical
\end{multicols*}
\newpage
\centering
{\fontsize{20.74}{70}\selectfont Emergency Procedures Checklist}
\break
\color{Red}
{\fontsize{10}{70}\selectfont For simulation use only, not for real world flight}
\color{Black}
\section*{LOW-VOLTAGE LIGHT ILLUMINATES DURING FLIGHT}
\centering \textbf{NOTE}:
Illumination of the low-voltage light may occur during low RPM conditions with an electrical load on the system such as during a low RPM taxi. Under these conditions, the light will go out at higher RPM. The master switch need not be recycled since an over-voltage condition has not occurred to de-activate the alternator system.\\
Radios\dotfill OFF\\
Alternator Circuit Breaker\dotfill CHECK IN\\
Master Switch\dotfill OFF (both sides)\\
Master Switch\dotfill ON\\
Low-Voltage Light\dotfill CHECK OFF\\
Radios\dotfill ON\\
If low-voltage light illuminates again:\\
Alternator\dotfill OFF\\
Nonessential Radio and Electrical Equipment\dotfill OFF\\
Flight\dotfill TERMINATE as soon as practical
\section*{INADVERTENT ICING ENCOUNTER}
\begin{flushleft}
Turn pitot heat switch ON.\\
Turn back or change altitude to obtain an outside air temperature that is less conducive to icing.\\
Pull cabin heat control full out and open defroster outlet to obtain maximum defroster air temperature. For greater air flow at reduced temperatures, adjust the cabin air control as required.\\
Open the throttle to increase engine speed and minimize ice buildup on propeller blades.\\
Watch for signs of carburetor air filter ice and apply carburetor as required. An unexplained loss in engine speed could be caused by carburetor ice or air intake filter ice. Lean the mixture for maximum RPM, if carburetor heat is used continuously.\\
Plan a landing at the nearest airport. With an extremely rapid ice build-up, select a suitable "off airport" landing site.\\
With an ice accumulation of 1/4 inch or more on the wing leading edges, be prepared for significantly higher stall speed.\\
Leave wing flaps retracted. With a severe ice build-up on the horizontal tail, the change in wing wake airflow direction caused by wing flap extension could result in a loss of elevator effectiveness.\\
Open left window and, if practical, scrape ice from a portion of the windshield for visibility in the landing approach.\\
Perform a landing approach using a forward slip, if necessary, for improved visibility.\\
Approach at 65 to 75 KIAS depending upon the amount of the accumulation.\\
Perform a landing in level attitude.
\end{flushleft}
\end{document}