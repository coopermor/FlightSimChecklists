\documentclass{article}
\usepackage[margin=0.1in]{geometry}
\usepackage[explicit]{titlesec}
\usepackage{xcolor}
\usepackage{multicol}
\usepackage{mathabx}

\definecolor{Red}{rgb}{1,0,0}
\definecolor{Orange}{rgb}{.92,.57,.12}
\definecolor{Black}{rgb}{0,0,0}
\definecolor{White}{rgb}{1,1,1}
\definecolor{Green}{rgb}{.3,.73,.16}
\titleformat{\section}
{\normalfont\normalsize\bfseries}{}{0em}{\colorbox{black}{\parbox{\dimexpr\linewidth-2\fboxsep\relax}{\textcolor{white}{\thesection\quad#1}}}\vspace{-1em}}
\titleformat{name=\section,numberless}
{\normalfont\normalsize\bfseries}{}{0em}{\colorbox{black}{\parbox{\dimexpr\linewidth-2\fboxsep\relax}{\textcolor{white}{#1}}}\vspace{-1em}}
\newcommand{\fly}{\vspace{-1em}\begin{center}\color{Green}FLY THE AIRPLANE\end{center}\vspace{-1em}}
\newcommand{\warning}[1]{\colorbox{Black}{\color{Orange}{#1}}}
\newcommand{\caution}[1]{\colorbox{Black}{\color{Red}{#1}}}
\usepackage{titlesec}
\titlespacing*{\section}{0pt}{0.1\baselineskip}{\baselineskip}

\renewcommand{\familydefault}{\sfdefault}
\begin{document}
\begin{center}
	{\fontsize{25}{70}\selectfont Daher TBM 930}
    \break
    {\fontsize{20.74}{70}\selectfont Emergency Procedures Checklist}
    \break
    \color{Red}
    {\fontsize{10}{70}\selectfont For simulation use only, not for real world flight}
\end{center}
\vspace{-2em}
\color{Black}
\begin{multicols*}{2}
\section*{ENGINE FAILURE AT TAKEOFF BEFORE ROTATION}
Throttle\dotfill Flight IDLE\\
Braking\dotfill As required\\
Throttle\dotfill CUT OFF\\
Tank Selector\dotfill OFF\\
Crash lever\dotfill PULL DOWN
\section*{OTHER REJECTED TAKEOFF}
Throttle\dotfill Flight IDLE\\
Reverse\dotfill As required\\
Braking\dotfill As required\\
Throttle\dotfill CUT OFF\\
Tank Selector\dotfill OFF\\
Crash lever\dotfill PULL DOWN
\section*{ENGINE FAILURE}
"MAN OVRD" control\dotfill FULL FORWARD\\
If successful: Fly the airplane using the "MAN OVRD" control for power, set throttle to Flight IDLE and land as soon as possible\\
If unsuccessful:\\
\hspace*{6mm}"MAN OVRD" control\dotfill FULL BACKWARD\\
If HEIGHT does not allow to choose a favorable runway or field, land straight ahead without changing landing gear position.\\
Flaps\dotfill "TO" (Maintain IAS $>$ 100 KIAS)\\
Throttle\dotfill "CUT OFF"\\
Tank selector\dotfill "OFF"\\
Before touchdown:
Flaps\dotfill "LDG"\\
Crash lever\dotfill PULL DOWN\\
If HEIGHT allows to reach a favorable runway:\\
Landing gear control\dotfill "DN"\\
Flaps\dotfill AS REQUIRED\\
Throttle\dotfill "CUT OFF"\\
Tank selector\dotfill "OFF"\\
Crash lever\dotfill PULL DOWN
\section*{ENGINE FAILURE IN FLIGHT}
Autopilot\dotfill DISCONNECT\\
Tank selector\dotfill SWITCH TANKS\\
"AUX BP" switch\dotfill CHECK / CORRECT\\
If successful: Check remaining fuel and land as soon as possible\\
In unsuccessful:\\
Throttle\dotfill "CUT OFF"\\
Oxygen mask\dotfill USE\\
\vspace{-1em}
\section*{\color{Red}OIL PRESS \color{White} OR \color{Orange}OIL PRESS}
\vspace{1em}\fly
Land as soon as possible\\
Monitor the oil pressure\\
Torque\dotfill MINIMUM NECESSARY\\
If engine looses power:
Throttle\dotfill "CUT OFF"\\
Perform a forced landing
\section*{\color{Orange}OIL TEMP}
With or without: RED WARNING CAS MESSAGE \warning{OIL PRESS} ON\\
Oil temperature indicator\dotfill CHECK\\
If the indicated temperature is in the green sector:\\
Land as soon as possible
\fly
Monitor\\
If the indicated temperature is not in the green sector:\\
Failure is confirmed, you can expect an OIL PRESSURE failure shortly\\
If engine looses power:\\
Throttle\dotfill "CUT OFF"\\
Perform a FORCED LANDING
\vfill\null
\section*{ENGINE REGULATION DISCREPANCY, POWER LOSS, THROTTLE CONTROL LOSS}
Throttle\dotfill Flight IDLE\\
Confirm engine still running\\
Tank selector\dotfill SWITCH TANKS\\
Check the no parameter exceeds allowed values\\
"MAN OVRD" control\dotfill ACTUATED progressively to MINIMUM NECESSARY\\
Continue flight, land as soon as possible\\
Perform a normal landing without reverse\\
Braking\dotfill AS REQUIRED\\
If minimum power obtained is excessive:\\
Reduce airspeed by setting airplane in nose-up attitude at IAS $<$ 178 KIAS\\
"INERT SEP" switch\dotfill "ON"\\
IF ITT $>$ 840° C: "INERT SEP" switch\dotfill "OFF"\\
Landing Gear control\dotfill "DN"\\
Flaps\dotfill "TO"\\
Establish a long final or an ILS approach respecting IAS $<$ 178 KIAS\\
When runway is assured: Fuel tank selector\dotfill "OFF"\\
Throttle\dotfill "FEATHER" (If available and necessary to extend trajectory)\\
Flaps\dotfill "LDG" as required (at IAS $<$ 122 KIAS)\\
Land normally WITHOUT REVERSE\\
Braking\dotfill AS REQUIRED
\section*{GOVERNOR CONTROL NOT OPERATING}
Continue the flight\\
If N\textsubscript{P} $<$ 1960 RPM, do not perform a go-around and do not use the reverse
\section*{EXCESSIVE PROPELLER ROTATION SPEED}
Reduce the power and the airplane speed to avoid propeller rotation speeds higher than 2000 RPM\\
Land as soon as possible\\
Do not perform ago-around
\section*{ENGINE DOES NOT STOP ON GROUND}
Tank selector\dotfill "OFF"\\
Wait for engine to sop due to lack of fuel in the pipes\\
"GENERATOR" selector\dotfill "OFF"\\
"SOURCE" selector\dotfill "OFF"\\
Crash lever\dotfill PULL DOWN
\section*{\color{Red}ITT \color{White}DURING ENGINE START}
Starting procedure\dotfill ABORT
\section*{\color{Red}ITT \color{White}AFTER ENGINE START}
\vspace{1em}
\fly
REDUCE POWER to have ITT $<$ 840° C\\
LAND AS SOON AS POSSIBLE
\section*{\color{Orange}CHIP}
LAND AS SOON AS PRACTICAL
\fly
Or DO NOT TAKE OFF\dotfill airplane is grounded
\vfill\null
\section*{AIR START PROCEDURES}
Oxygen mask\dotfill USE\\
"GENERATOR" Switch\dotfill "MAIN"\\
"BLEED" Switch\dotfill "OFF"\\
"A/C" Switch\dotfill "OFF"\\
Electric Consumption\dotfill Reduce\\
Tank selector\dotfill "L" or "R" checked\\
"AUX BP" fuel switch\dotfill "L" or "R" checked\\
"IGNITION" switch\dotfill "AUTO" or "ON"\\
Throttle\dotfill  "CUT OFF"\\
"STARTER" switch "ON", start timer\\
\underline{When Ng around 13\%:}\\
Throttle\dotfill LO / IDLE\\
ITT and Ng\dotfill Monitor\\
\underline{When Ng around 52\%:}\\
Check starter is\dotfill "OFF" automatically\\
Throttle\dotfill FLIGHT IDLE\\
Throttle\dotfill As required\\
Electrical Equipment\dotfill As required\\
"AUX BP" Switch\dotfill "AUTO"\\
"BLEED" switch\dotfill As required\\
If necessary\dotfill EMERGENCY DESCENT\\
If AIR START not successful\dotfill FORCED LANDING
\section*{ENGINE FIRE ON GROUND}
Throttle\dotfill "CUT OFF"\\
"BLEED" switch\dotfill "OFF"\\
"A/C" switch\dotfill "OFF"\\
Brakes\dotfill AS REQUIRED\\
Tank Selector\dotfill "OFF"\\
Crash lever\dotfill PULL DOWN\\
EVACUATE
\section*{CABIN FIRE ON GROUND}
Throttle\dotfill "CUT OFF"\\
Crash lever\dotfill PULL DOWN\\
Cabin extinguisher\dotfill AS REQUIRED\\
EVACUATE
\section*{ENGINE FIRE IN FLIGHT}
\textbf{\underline{CAUTION}: \\NO AIR START ATTEMPT AFTER AN ENGINE FIRE}
\fly
Throttle\dotfill "CUT OFF"\\
"AUX BP" fuel switch\dotfill "OFF"\\
Tank Selector\dotfill "OFF"\\
Oxygen mask\dotfill USE\\
"BLEED" switch\dotfill "OFF"\\
"A/C" switch\dotfill "OFF"\\
If necessary\dotfill EMERGENCY DESCENT
\section*{CABIN ELECTRICAL FIRE OR SMOKE DURING FLIGHT}
\vspace{1em}
\fly
OXYGEN mask and GOGGLES\dotfill USE\\
If the origin is known:\\
Defective equipment circuit breaker\dotfill PULL\\
Extinguisher\dotfill USE\\
If the origin is unknown:\\
"A/C" switch\dotfill "OFF"\\
All unnecessary equipment\dotfill OFF\\
Perform\dotfill EMERGENCY DESCENT\\
If necessary\dotfill SMOKE ELIMINATION\\
LAND as soon as possible
\vfill\null
\section*{SMOKE ELIMINATION}
OXYGEN mask and GOGGLES\dotfill USE\\
"BLEED" switch "OFF"\\
"A/C" switch\dotfill "OFF"\\
"DUMP" switch\dotfill ACTUATE\\
Wait until the differential pressure drops\\
"RAM AIR" control knob\dotfill PULL\\
If smoke increases\dotfill PUSH\\
LAND as soon as possible
\section*{MAXIMUM RATE DESCENT}
Throttle\dotfill Flight IDLE\\
Oxygen mask\dotfill USE\\
Pitch attitude -10° to -20°\\
If smooth air:
Flaps and Landing gear control\dotfill "UP"\\
Speed\dotfill VMO = 266 KIAS\\
If rough air or in case of structure problem:\\
Speed\dotfill BELOW 178 KIAS\\
Flaps\dotfill "UP"\\
Landing gear control\dotfill "DN"
\section*{MAXIMUM RANGE DESCENT}
Oxygen mask\dotfill USE\\
Throttle\dotfill "CUT OFF"\\
Flaps and Landing gear control\dotfill "UP"\\
Speed\dotfill 120 KIAS\\
"DUMP" switch\dotfill ACTUATE\\
"RAM AIR" control knob\dotfill PULL\\
If VMC and non icing conditions are possible:\\
"ESS BUS TIE" switch\dotfill Cover up then "EMER" position\\
Prepare for\dotfill FORCED LANDING\\
If VMC and non icing conditions are not possible\\
All DE-ICE switches\dotfill "OFF"\\
All light switches\dotfill "OFF"\\
"BLEED" switch\dotfill "OFF"\\
"A/C" switch\dotfill "OFF"\\
"AUX BP" switch\dotfill "OFF"\\
"FUEL SEL" switch\dotfill "MAN"\\
"AP / TRIMS" switch\dotfill "OFF"\\
"PFD 2" breaker\dotfill PULL\\
"ADC 2" breaker\dotfill PULL\\
"XPDR 2" breaker\dotfill PULL\\
If icing conditions:\\
\hspace*{6mm}"PITOT L HTR" switch\dotfill "ON"\\
\hspace*{6mm}"WINDSHIELD" switch\dotfill ON\\
Speed\dotfill ABOVE 135 KIAS\\
If time permits:\\
\hspace*{6mm}"PLUGS" breaker\dotfill PULL\\
\hspace*{6mm}"AIR COND" breaker\\
Prepare a forced landing
\vfill\null
\section*{FORCED LANDINGS}
Throttle\dotfill "CUT OFF"\\
Tank selector\dotfill "OFF"\\
"AUX BP" switch\dotfill "OFF"\\
"BLEED" and "A/C" switches\dotfill "OFF"\\
"DUMP" switch\dotfill ACTUATE\\
Maintain 120 KIAS of gliding speed until favorable ground approach\\
If ground allows it:\\
\hspace*{6mm} "ESS BUS TIE" switch\dotfill "NORM" (to have GEAR and FLAPS available)\\
\hspace*{6mm} Landing gear control\dotfill "DN"\\
If night conditions:\\
\hspace*{6mm} Lights\dotfill "LDG"\\
If ground does not allow it:\\
\hspace*{6mm} Landing gear\dotfill Keep "UP"\\
\hspace*{6mm} Flaps\dotfill "LDG" (when chosen ground is assured)\\
Crash lever\dotfill PULL DOWN\\
Speed on final approach\dotfill 85 KIAS\\
Land flaring out\\
EVACUATE after stop
\section*{TIRE BLOWOUT DURING LANDING}
Control direction with brakes and nose wheel steering\\
REVERSE\dotfill AS REQUIRED\\
Stop airplane to minimize damages\\
Perform ENGINE SHUT-DOWN
\section*{FLAPS MALFUNCTION}
FLAPS circuit breaker\dotfill PULL\\
Flap contorl lever\dotfill "UP"\\
Land as soon as possible maintaining airspeeds:\\
\hspace*{6mm} IAS $\leq$ 178 KIAS for deflections between "UP" and "TO" positions\\
\hspace*{6mm} IAS $\leq$ 122 KIAS for deflections greater than "TO" position\\
For landing, refer to "LANDING WITH FLAPS MALFUNCTION"
\section*{LANDING WITH FLAPS MALFUNCTION}
For flaps deflections between "UP" and "TO":
\hspace*{6mm} Proceed as for a normal landing with 105 KIAS of approach speed\\
\hspace*{6mm} Provide for a landing distance increased by 60\%\\
For flaps deflections greater than "TO":
\hspace*{6mm} Proceed as for a normal landing with 100 KIAS of approach speed\\
\hspace*{6mm} Provide for a landing distance increased by 50\%
\section*{LANDING GEAR RETRACTION DISCREPANCY}
\warning{GEAR UNSAFE} CAS message and "GEAR UNSAFE" red warning light ON or amber light flashing and 3 green lights OFF\\
Maintain IAS below 150 KIAS\\
"LDG GEAR" circuit breaker\dotfill PULL\\
If \warning{GEAR UNSAFE} CAS message and "GEAR UNSAFE" red warning light are off:
\hspace*{6mm} The flight may be continued without any restriction\\
If not:\\
\hspace*{6mm} "LDG GEAR" circuit breaker\dotfill PUSH\\
Refer to "EMERGENCY GEAR EXTENSION"
\section*{LANDING GEAR EXTENSION DISCREPANCY}
\warning{GEAR UNSAFE} CAS message and "GEAR UNSAFE" red warning light ON or amber light flashing and 3 green lights OFF\\
Maintain IAS below 150 KIAS\\
Refer to "EMERGENCY GEAR EXTENSION"
\vfill\null
\section*{EMERGENCY GEAR EXTENSION}
Maintain IAS below 150 KIAS\\
Landing gear control\dotfill "DN"\\
"LDG GEAR" circuit breaker\dotfill PULL\\
Floor hatch\dotfill OPEN\\
By-pass selector\dotfill FULLY PULL / LOCKED\\
Hand pump\dotfill ACTUATE with maximum amplitude\\
Press the CAS MASTER WARNING push-button to reset the \warning{GEAR UNSAFE} CAS message\\
If "GEAR UNSAFE" red warning light is not illuminated and 3 green lights are illuminated:\\
\hspace*{6mm} Continue flight if necessary BELOW 178 KIAS, exit and/or remaining outside icing conditions
\section*{LANDING WITH UNLOCKED MAIN LANDING GEAR}
"BLEED" switch\dotfill OFF\\
"DUMP" switch\dotfill ACTUATED\\
Maintain tank selector on defective landing gear side to lighten corresponding wing [maximum fuel unbalance 15 USG (57 liters)]\\
Choose a runway with headwind or crosswind blowing from defective gear side\\
Align the airplane to land on the runway edge opposite to the defective landing gear\\
Do a normal approach at 90 KIAS, flaps on "LDG"\\
Land and set nose gear immediately on ground to assure lateral control\\
Use full aileron during roll-out to lift the wing with the defective landing gear\\
Preferably do not use reverse\\
Complete taxiing with a slight turn toward defective landing gear\\
Throttle\dotfill "CUT OFF"\\
Engine stop procedure\dotfill COMPLETE\\
EVACUATE\\
If landing gear drags during landing:\\
\hspace*{6mm} Throttle\dotfill "CUT OFF"\\
\hspace*{6mm} Crash lever\dotfill PULL DOWN\\
\hspace*{6mm} Tank selector\dotfill "OFF"\\
\hspace*{6mm} EVACUATE after airplane comes to a stop
\section*{LANDING WITH DEFECTIVE NOSE LANDING GEAR (DOWN UNLOCKED OR NOT DOWN)}
Approach\dotfill Flaps "LDG"\\
Airspeed\dotfill 90 KIAS\\
Land with nose-up attitude, keep nose high\\
Throttle\dotfill "CUT OFF"\\
Touch-down slowly with nose wheel and keep elevator at nose-up stop\\
Moderate braking\\
Crash lever\dotfill PULL DOWN\\
EVACUATE after airplane comes to a stop
\section*{LANDING WITH GEAR UP}
Final approach\dotfill Standard\\
Flaps\dotfill "LDG"\\
Airspeed\dotfill 85 KIAS\\
"BLEED" switch\dotfill "OFF"\\
"DUMP" switch\dotfill ACTUATE\\
When runway is assured:\\
Throttle\dotfill "CUT OFF"\\
Tank selector\dotfill "OFF"\\
Flare out\\
After touch-down, crash lever\dotfill PULL DOWN\\
EVACUATE after airplane comes to a stop
\vfill\null
\section*{DITCHING}
Landing gear\dotfill "UP"\\
In heavy swell with light wind, land parallel to the swell (rollers).\\
In heavy wind, land facing wind.\\
Flaps\dotfill "LDG"\\
Maintain a descent rate as low as possible when approaching the water\\
Airspeed\dotfill IAS$\geq$85 KIAS\\
"BLEED" switch\dotfill "OFF"\\
"DUMP" switch\dotfill ACTUATE\\
Crash lever\dotfill PULL DOWN\\
Maintain attitude without rounding off until touch-down\\
EVACUATE through EMERGENCY EXIT
\section*{LANDING WITHOUT ELEVATOR CONTROL}
Landing gear\dotfill "DN"\\
Flaps\dotfill "LDG"\\
Power as necessary to maintain airspeed according to an easy approach slope $\approx$300 ft / min\\
Adjust elevator by using manual pitch trim wheel\\
When ground approaches, decrease slope progressively\\
Reduce power progressively\\
\section*{\color{Red}FLAPS ASYM}
\vspace{1em}
\fly
FLAPS circuit breaker\dotfill PULL\\
FLAPS control lever\dotfill "UP"\\
LAND as soon as possible maintaining airspeeds:
\hspace*{6mm}IAS $\leq$ 178 KIAS for deflections between "UP" and "TO" positions\\
\hspace*{6mm}IAS $\leq$ 122 KIAS for deflections greater than "TO" position\\
For landing, refer to "LANDING WITH FLAPS MALFUNCTION"
\section*{\color{Red}FUEL PRESS}
\vspace{1em}
\fly
Remaining fuel\dotfill CHECK\\
Tank selector\dotfill SWITCH TANKS\\
"AUX BP" fuel switch\dotfill "AUTO"
If \warning{FUEL PRESS} remains ON\\
"AUX BP" fuel switch\dotfill "ON"\\
Check message \caution{AUX BOOST PMP ON}\\
Maintain "AUX BP" fuel switch\dotfill "ON"\\
{\color{Green}LAND AS SOON AS PRACTICAL}\\
If \warning{FUEL PRESS} remains ON\\
\hspace*{6mm} Tank selector\dotfill SWITCH TANKS\\
\warning{FUEL PRESS} is OFF, a supply problem may have occurred from the tank selected first\\
If \warning{FUEL PRESS} remains ON\\
\hspace*{6mm} Fullest tank\dotfill SELECT\\
Avoid high power and rapid movements of the throttle\\
Descent to an altitude below 18000 ft\\
LAND as soon as possible
\fly

\section*{\color{Orange}AUX BOOST PMP ON}
\vspace*{1em}\fly
If "AUX BP" fuel switch is in "AUTO" position:\\
\hspace*{6mm} RESET to\dotfill "ON"\\
\hspace*{6mm} THEN to "AUTO"\\
If \warning{AUX BOOST PUMP ON} \underline{GOES OFF:}
\vspace{-1em}
\begin{center}
    \color{Green}Continue the flight normally
\end{center}
\vspace{-.5em}
If \warning{AUX BOOST PUMP ON} \underline{remains ON,} mechanical booster pump has failed\\
\hspace*{6mm} "AUX BP" fuel switch\dotfill "ON"\\
\hspace*{6mm} LAND AS SOON AS POSSIBLE
\vfill\null
\section*{\color{Orange}FUEL LOW L-R}
Corresponding gage\dotfill CHECK\\
Check the other tank has not been automatically selected\\
If not:\\
\hspace*{6mm}  "FUEL SEL" switch \dotfill "MAN"\\
\hspace*{6mm} Select fuel tank manually\dotfill as required
\begin{center}
    \color{Green}FLY THE AIRPLANE\\CHECK MINIMUM FUEL\\
    \color{Black}TAKE DECISION, land as soon as practical if necessary
\end{center}
\vspace{-1em}
\section*{\color{Orange}AUTO SEL}
\vspace{1em}\fly
"FUEL SEL" switch\dotfill "AUTO"\\
If it is on "AUTO", failure is confirmed\\
"FUEL SEL" switch\dotfill "MAN"\\
Select tanks manually as required\\
\textbf{\underline{CAUTION:} MAXIMUM UNBALANCE IS 15 USG}
\section*{\color{Orange}FUEL IMBALANCE}
If "FUEL SEL" on "AUTO" mode\\
\hspace*{6mm} SELECT the fullest tank\dotfill by pressing the "SHIFT" push-button\\
If "FUEL SEL" on "MAN" mode\\
\hspace*{6mm} SELECT the fullest tank\dotfill by shifting the tank selector manually
\section*{LOW LVL FAIL L - R}
CHECK\dotfill Fuel remaining in tanks\\
MAKE DECISION\\
If any doubt\dotfill LAND AS SOON AS PRACTICAL
\vspace{-1em}
\fly
\vspace{-1em}
\section*{\color{Orange}BAT OFF}
"SOURCE" selector\dotfill "OFF"\\
"SOURCE" selector\dotfill "BATT"\\
If warning persists\dotfill Land as soon as possible\\
Monitor airplane mains voltage
\section*{\color{Orange}MAIN GEN}
If necessary\dotfill CORRECT\\
If warning persists\dotfill "MAIN GEN" switching confirmed\\
"MAIN GENERATOR RESET" push-button\dotfill PUSH
\fly
Keep the following systems connected:
\hspace*{6mm} A/P system
\hspace*{6mm} Deicing systems except right windshield
\hspace*{6mm} STROBE and NAV lights
\hspace*{6mm} Cockpit emergency lights
\hspace*{6mm} VHF 1
\hspace*{6mm} NAV/GPS 1
\hspace*{6mm} BLEED
\hspace*{6mm} Landing lights on short final
"GENERATOR" selector (RESET if necessary)\dotfill "ST-BY"\\
Maintain ST-BY loads below 100A
\section*{LOW VOLTAGE}
Voltmeter voltages\dotfill CHECK\\
If voltages are $<$ 26 Volts, monitor a possible drop or any indication of battery discharge\\
In that case:
\fly
Keep the following systems connected:\\
\hspace*{6mm} A/P system\\
\hspace*{6mm} Deicing systems except right windshield\\
\hspace*{6mm} STROBE and NAV lights\\
\hspace*{6mm} Cockpit emergency lights\\
\hspace*{6mm} VHF 1\\
\hspace*{6mm} NAV/GPS 1\\
\hspace*{6mm} BLEED\\
\hspace*{6mm} Landing lights on short final\\
"GENERATOR" selector (RESET if necessary)\dotfill "ST-BY"\\
Maintain ST-BY loads below 100A
\vfill\null
\section*{\color{Orange}MAIN GEN \color{White}AND \color{Orange}LOW VOLTAGE}
"GENERATOR" selector\dotfill "MAIN"\\
"MAIN GENERATOR RESET" push-button\dotfill PRESS\\
\vspace{-1em}\fly
If successful:\\
\hspace*{6mm} Disconnect ancillary electrical systems not essential\\
\hspace*{6mm} Monitor voltmeter and ammeter\\
Prepare to LAND AS SOON AS POSSIBLE\\
If not successful:\\
"GENERATOR" selector\dotfill "ST-BY"\\
"ST-BY GENERATOR RESET" push-button\dotfill PRESS\\
If successful:\\
\hspace*{6mm} Disconnect ancillary electrical systems not essential\\
\hspace*{6mm} Monitor voltmeter and ammeter\\
Prepare to LAND AS SOON AS POSSIBLE\\
If not successful, both generators failure is confirmed. If possible, return to VMC conditions\\
"GENERATOR" selector\dotfill "OFF"\\
If conditions allow: VMC and non icing conditions\\
If altitude $\geq$ 10000 ft : "OXYGEN" switch\dotfill "ON"\\
"ESS BUSS TIE" switch\dotfill Cover up, then "EMER" position\\
In this configuration, only both "ESS BUS" bars and "BUS BATT" bar are directly supplied by the battery
LAND as soon as possible\\
If necessary, it is always possible to use other ancillary systems by selecting\\
"ESS BUSS TIE" switch\dotfill "NORM"\\
If conditions do not allow:\\
Manually disconnect ancillary systems as follows:\\
\hspace*{6mm} "AIRFRAME DE ICE" switch\dotfill "OFF"\\
\hspace*{6mm} "ICE LIGHT" switch\dotfill "OFF"\\
\hspace*{6mm} "PROP DE ICE" switch\dotfill "OFF"\\
\hspace*{6mm} "WINDSHIELD" switch\dotfill "OFF"\\
\hspace*{6mm} "PITOT R \& STALL HTR" switch\dotfill "OFF"\\
\hspace*{6mm} "OFF/LDG/TAXI" light"PULSE" switches\dotfill "OFF"\\
\hspace*{6mm} "STROBE" switch\dotfill "OFF"\\
\hspace*{6mm} "BLEED" and "A/C" switches\dotfill "OFF"\\
\hspace*{6mm} "AUX BP" switch\dotfill "OFF"\\
\hspace*{6mm} "FUEL SEL" switch\dotfill "MAN"\\
\hspace*{6mm} "AP / TRIMS" switch\dotfill "OFF"\\
\hspace*{6mm} "PFD 2" breaker\dotfill "PULL"\\
\hspace*{6mm} "ADC 2" breaker\dotfill "PULL"\\
\hspace*{6mm} "TAS" breaker\dotfill "PULL"\\
\hspace*{6mm} "DATA LINK" breaker\dotfill "PULL"\\
\hspace*{6mm} "DIMMER / CABIN / ACCESS" controls\dotfill "OFF"\\
\hspace*{6mm} "XPDR 2" breaker\dotfill "PULL"\\
If icing conditions:\\
\hspace*{6mm} "PITOT L HTR" switch\dotfill Checked "ON"\\
\hspace*{6mm} "WINDSHIELD" switch\dotfill "ON"\\
Maintain minimum recommended speeds into known icing conditions\\
\begin{tabular}{|l|l|}
    \hline
    Flaps UP  & 135 KIAS \\ \hline
    Flaps TO  & 110 KIAS \\ \hline
    Flaps LDG & 90 KIAS  \\ \hline
    \end{tabular}
\break
If time permits:\\
\hspace*{6mm} "PLUGS" breakers\dotfill PULL\\
\hspace*{6mm} "AIR COND" breaker\dotfill PULL\\
LAND as soon as possible
\section*{\color{Red}{ELEC FEATH FAULT}}
\vspace{1em}
\fly
"FEATHER" circuit breaker\dotfill PULL\\
LAND as soon as possible
\vfill\null
\section*{TOTAL LOSS OF ELECTRICAL POWER}
Maintain airplane control\\
Use the MD 302 for\dotfill attitude, airspeed and/or altitude
\fly
Land as soon as possible
\section*{\color{Red}{BLEED TEMP}}
\vspace{1em}
\fly
Should automatic cut off occur or not:\\
If possible\dotfill REDUCED POWER\\
"HOT AIR FLOW" distributor\dotfill turn to the right\\
"A/C" switch\dotfill "PILOT"\\
"TEMP" selector\dotfill Mini\\
"BLEED" switch\dotfill "OFF"\\
"BLEED" switch\dotfill "AUTO"\\
If \caution{BLEED TEMP} and \warning{BLEED OFF} warnings till ON:\\
\hspace*{6mm} Refer to "BLEED OFF"\\
If \caution{BLEED TEMP} ON (No \warning{BLEED OFF}):\\
\hspace*{6mm}Shorten the flight
\section*{\color{Orange}{BLEED OFF}}
USE OXYGEN MASK\\
Check "BLEED" switch position and\dotfill CORRECT\\
\vspace{-1em}
If possible, reduce power\\
\fly
"BLEED" switch\dotfill "OFF"\\
"BLEED" switch\dotfill "AUTO"\\
If in flight:\\
\hspace*{6mm} If warning \warning{BLEED OFF} still displayed\\
\hspace*{6mm} If necessary\dotfill EMERGENCY DESCENT\\
\hspace*{6mm} Continue flight\\
If on the ground:\\
\hspace*{6mm} "BLEED" switch\dotfill "OFF"\\
\hspace*{6mm} Taxi back to the apron\\
\hspace*{6mm} Normal engine shut-down
\section*{\color{Red}{CABIN ALTITUDE} AND \warning{USE OXYGEN MASK}}
Pressurization indicator\dotfill CHECK\\
If cabin altitude $>$ 10000 ft:\\
\vspace{-1em}
OXYGEN\dotfill  USE OXYGEN MASK\\
\fly
"BLEED" switch\dotfill CHECK "AUTO"\\
"DUMP" switch\dotfill CHECK UNDER GUARD\\
"EMERGENCY RAM AIR" control knob\dotfill CHECK PUSHED\\
If necessary\dotfill EMERGENCY DESCENT\\
\section*{\color{Red}{CABIN ALTITUDE} AND \warning{USE OXYGEN MASK} AND \caution{EDM}}
Pressurization indicator\dotfill CHECK\\
If cabin altitude $>$ 10000 ft:\\
\vspace{-.8em}
OXYGEN\dotfill  USE OXYGEN MASK\\
\fly
\vspace{.2em}
"BLEED" switch\dotfill CHECK "AUTO"\\
"DUMP" switch\dotfill CHECK UNDER GUARD\\
"EMERGENCY RAM AIR" control knob\dotfill CHECK PUSHED\\
If necessary\dotfill EMERGENCY DESCENT\\
\section*{\color{Red}{CABIN DIFF PRESS}}
Pressurization indicator\dotfill CHECK\\
If $\Delta$ 6.4$\pm$ 0.2 PSI:\\
\hspace*{6mm} "BLEED" switch\dotfill "OFF"\\
\hspace*{6mm} OXYGEN\dotfill USE, if necessary\\
\vspace{-.8em}
\fly
\vspace{.4em}
If necessary (no oxygen available)\dotfill EMERGENCY DESCENT\\
\section*{CABIN NOT DEPRESSURIZED AFTER LANDING}
"DUMP" switch\dotfill ACTUATED\\
"BLEED" switch\dotfill "OFF"\\
"EMERGENCY RAM AIR" control knob\dotfill PULLED if necessary\\
Wait for complete cabin depressurization before opening the door
\vfill\null
\section*{VACUUM LOW}
\textbf{MONITOR}\\
If necessary, fly to an altitude $\leq$ 10000 ft and return to VMC conditions as soon as
possible.
\fly
"BLEED" switch\dotfill "OFF"
\section*{LEADING EDGES DEICING FAILURE}
LEAVE icing conditions as soon as possible\\
"AIRFRAME DE ICE" switch\dotfill "OFF"
\section*{\color{Orange}{PROP DEICE FAIL}}
REDUCE power
\fly
ACTUATE throttle\dotfill to vary RPM within operating range\\
LEAVE icing conditions\dotfill as soon as possible
\section*{\color{Orange}{INERT SEP FAIL}}
LEAVE icing conditions\dotfill as soon as possible
\fly
\section*{WINDSHIELD DEICING FAILURE}
"WINDSHIELD" switch \dotfill "OFF" / "ON" when necessary\\
In case of total failure:\\
"TEMP" selector\dotfill Max warm\\
"HOT AIR FLOW" distributor\dotfill turn to the left\\
Before landing wait for a sufficient visibility
\section*{WINDSHIELD MISTING OR INTERNAL ICING}
"TEMP" selector\dotfill Set to 12 o'clock position\\
"HOT AIR FLOW" distributor\dotfill turn to the left\\
"WINDSHIELD" switch\dotfill "ON"\\
If not successful, to gain sufficient visibility:\\
"HOT AIR FLOW" distributor\dotfill fully turn to the left\\
Manually clean a sufficient visibility area.\\
If necessary, clean L.H. side window and conduct a sideslip approach (rudder pedals to the right) in order to get sufficient landing visual references.\\
For landing with flaps "LDG", maintain\dotfill IAS $\geq$ 95 KIAS
\section*{\color{Orange}{PITOT NO HT L-R}}
\warning{PITOT NO HT L}\\
Avoid icing conditions
\fly
If it is not possible:\\
Perform moderate descent or climb attitudes, V\textsubscript{MO} overshoot and stall warning systems are always operating\\
\warning{PITOT NO HT R}\\
V\textsubscript{MO} overshoot warning may be altered by icing conditions

\fly
Monitor maximum airspeed\dotfill $\leq$ 266 KIAS
\section*{STALL NO HEAT}
MONITOR and MAINTAIN minimum airspeed according to airplane configuration and icing conditions
\fly
\section*{RUNAWAY OF TRIM}
\vspace{1em}
\fly
"AP / TRIM DISC" push-button\dotfill PUSHED AND \underline{HELD}\\
"AP / TRIMS" switch\dotfill RELEASED\\
Pitch trim may be used manually
Reduce airspeed if necessary to reduce control forces
If pitch trim runaway
"AP / TRIMS" switch\dotfill "AP OFF"\\
If rudder or aileron trim runaway\\
PULL circuit breaker\dotfill corresponding to the defective trim tab\\
"AP / TRIMS" switch\dotfill "ON"\\
\section*{CRACK IN COCKPIT WINDOW OR WINDOW PANEL}
\vspace{1em}
\fly
DESCEND SLOWLY\\
Reduce cabin $\Delta$P\dotfill by setting Landing Field Elevation to 10000 ft\\
\vfill\null
\section*{EMERGENCY EXIT USE}
Check that the anti-theft safety pin has been removed\\
Lift up the opening handle\\
Pull emergency exit assembly toward oneself to release it from its recess\\
Put the emergency exit door inside fuselage or throw it away from the fuselage through the opening\\
EVACUATE airplane
\section*{EMERGENCY BEACON (ELT) USE}
On COM VHF 121.5 MHZ or on a known air traffic control frequency, transmit the "MAY DAY" signal if possible\\
After landing:\\
\hspace*{6mm} "ELT" remote control switch\dotfill "ON" (maintain it "ON" until aid arrives)
\section*{INADVERTENT SPINS}
"AP / TRIMS DISC" push-button\dotfill PRESS and HOLD until recovery\\
Control wheel\dotfill NEUTRAL : PITCH ROLL\\
Rudder\dotfill FULLY OPPOSED TO THE SPIN\\
Throttle\dotfill FLIGHT IDLE\\
Flaps\dotfill "UP"\\
When rotation is stopped\\
Level the winds and ease out of the dive
\fly
\section*{AP OFF, STALL WARNING}
Fly the airplane, wings level and nose down until stall warning stops\\
Power as required\\
Return to the desired flight path
\section*{\color{Red}USP ACTIVE}
Do not disconnect AP\\
Increase power up to 50 \% minimum\\
Manage the flight
\section*{AIRSPEED INDICATING SYSTEM FAILURE}
"PITOT L HTR" switch\dotfill CHECK "ON"\\
"PITOT R \& STALL HTR" switch\dotfill CHECK "ON"\\
If symptoms persist:\\
"ALTERNATE STATIC" selector\dotfill PULL THOROUGHLY
\section*{\color{Orange}IGNITION}
CHECK\dotfill "IGNITION" switch position\\
If weather permits\dotfill correct by switching to "AUTO"
\fly
\section*{AUTOPILOT OR ELECTRIC PITCH TRIM MALFUNCTION}
"AP / TRIM DISC" push-button\dotfill PRESSED and HELD\\
"AP / TRIMS" switch\dotfill OFF\\
"AP / TRIM DISC" push-button\dotfill RELEASED\\
If necessary, control wheel\dotfill RETRIM
\section*{MFD FAILURE}
PFD1 display back-up button\dotfill Pressed\\
MFD circuit breaker\dotfill Checked IN\\
\end{multicols*}
\end{document}