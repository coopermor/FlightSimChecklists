\documentclass{article}
\usepackage[margin=0.1in]{geometry}
\usepackage[explicit]{titlesec}
\usepackage{multicol}
\usepackage{multirow}
\usepackage[table,xcdraw]{xcolor}
\definecolor{Red}{rgb}{1,0,0}
\definecolor{Black}{rgb}{0,0,0}
\titleformat{\section}
{\normalfont\normalsize\bfseries}{}{0em}{\colorbox{black}{\parbox{\dimexpr\linewidth-2\fboxsep\relax}{\textcolor{white}{\thesection\quad#1}}}\vspace{-1em}}
\titleformat{name=\section,numberless}
{\normalfont\normalsize\bfseries}{}{0em}{\colorbox{black}{\parbox{\dimexpr\linewidth-2\fboxsep\relax}{\textcolor{white}{#1}}}\vspace{-1em}}

\usepackage{titlesec}
\titlespacing*{\section}{0pt}{0.1\baselineskip}{\baselineskip}

\renewcommand{\familydefault}{\sfdefault}
\begin{document}
\begin{center}
    {\fontsize{25}{70}\selectfont Cirrus SR 22}
    \break
    {\fontsize{20.74}{70}\selectfont Quick Reference Checklist}
    \break
    \color{Red}
    {\fontsize{10}{70}\selectfont For simulation use only, not for real world flight}
\end{center}
\vspace{-2em}
\begin{multicols*}{2}
\section*{AIRSPEEDS FOR NORMAL OPERATION}
Takeoff Roataion:\\
\hspace*{6mm} Normal, Flaps 50\%\dotfill 70 KIAS\\
\hspace*{6mm} Obstacle Clearance, Flaps 50\%\dotfill 78 KIAS\\
Enroute Climb, Flaps Up:\\
\hspace*{6mm} Normal\dotfill 110-120 KIAS\\
\hspace*{6mm} Best Rate of Climb, SL\dotfill 101 KIAS\\
\hspace*{6mm} Best Rate of Climb, 10,000\dotfill 95 KIAS\\
\hspace*{6mm} Best Angle of Climb, SL\dotfill 78 KIAS\\
\hspace*{6mm} Best Angle of Climb, 10,000\dotfill 82 KIAS\\
Landing Approach:\\
\hspace*{6mm} Normal Approach, Flaps Up\dotfill 90-95 KIAS\\
\hspace*{6mm} Normal Approach, 50\%\dotfill 85-90 KIAS\\
\hspace*{6mm} Normal Approach, Flaps 100\%\dotfill 80-85 KIAS\\
\hspace*{6mm} Short Field, Flaps 100\% (V\textsubscript{REF})\dotfill 77 KIAS\\
Go-Around, Flaps 50\%:\\
\hspace*{6mm} Full Power\dotfill 80 KIAS\\
Maximum Recommended Turbulent Air Penetration:\\
\hspace*{6mm} 3400 lb\dotfill 133 KIAS\\
\hspace*{6mm} 2900 lb\dotfill 123 KIAS\\
Maximum Demonstrated Crosswind Velocity:\\
\hspace*{6mm} Takeoff of Landing\dotfill 20 Knots
\section*{PREFLIGHT}
Avionics Power Switch\dotfill OFF\\
Bat 2 Master Switch\dotfill ON\\
Avionics Cooling Fan\dotfill Audible\\
Voltmeter\dotfill 23-25 Volts\\
Flap Postion Light\dotfill OUT\\
Bat 1 Master Switch\dotfill ON\\
Fuel Quantity\dotfill Check\\
Fuel Selector\dotfill Select Fullest Tank\\
Flaps\dotfill 100\%, Check Light ON\\
Oil Annunciator\dotfill ON\\
Lights\dotfill Check Operation\\
Bat 1 and 2 Master Switches\dotfill OFF\\
Alternate Static Source\dotfill NORMAL\\
Circuit Breakers\dotfill IN\\
Elevator\dotfill Movememnt\\
Rudder\dotfill Movement\\
Ailerons\dotfill Movement
\section*{BEFORE STARTING ENGINE}
Preflight Inspection\dotfill COMPLETED
\section*{STARTING ENGINE}
External Power (If applicable)\dotfill CONNECT\\
Brakes\dotfill HOLD\\
Bat Master Switches\dotfill ON (Check Volts)\\
Strobe Lights\dotfill ON\\
Mixture\dotfill FULL RICH\\
Power Lever\dotfill FULL FORWARD\\
Fuel Pump\dotfill PRIME, then BOOST\\
Propeller Area\dotfill CLEAR\\
Power Lever\dotfill OPEN 1/4 INCH\\
Ignition Switch\dotfill START (Release after engine starts)\\
Power Lever\dotfill RETARD (to maintain 1000 RPM)\\
Oil Pressure\dotfill CHECK\\
Alt Master Switches\dotfill ON\\
Avionics Power Switch\dotfill ON\\
Engine Paramaters\dotfill MONITOR\\
External Power (If applicable)\dotfill DISCONNECT\\
Ammeter\dotfill CHECK
\section*{BEFORE TAKEOFF}
Brakes\dotfill CHECK\\
Flaps\dotfill UP (0\%)\\
Radios/Avionics\dotfill AS REQUIRED\\
Cabin Heat/Defrost\dotfill AS REQUIRED
\section*{TAXIING}
HSI Orentation\dotfill CHECK\\
Attitude Gyro\dotfill CHECK\\
Turn Coordinator\dotfill CHECK
\section*{BEFORE TAKEOFF}
Brakes\dotfill HOLD\\
Flight Controls\dotfill FREE \& Correct\\
Trim\dotfill SET Takeoff\\
Autopilot\dotfill DISCONNECT\\
Flaps\dotfill SET 50\% \& CHECK\\
Flight and Engine Instruments\dotfill CHECK\\
HSI and Altimeter\dotfill CHECK \& SET\\
Fuel Quantity\dotfill CONFIRM\\
Fuel Selector\dotfill FUELLEST TANK\\
Propeller\dotfill CHECK\\
\hspace*{6mm} Power Lever\dotfill INCREASE to detent\\
\hspace*{6mm} Note RPM rises to approximately 2000 RPM and manifold pressure increases slightly as Power Lever is set in dentent\\
\hspace*{6mm} Power Lever\dotfill 1700 RPM\\
Alternator\dotfill CHECK\\
\hspace*{6mm} Pitot Heat\dotfill ON\\
\hspace*{6mm} Avionics\dotfill ON\\
\hspace*{6mm} Navigation Lights\dotfill ON\\
\hspace*{6mm} Landing Light\dotfill ON (3-5 seconds)\\
\hspace*{6mm} Verify both ALT 1 and ALT 2 caution lights out and postive amps indication for each alternator. If necessary, increase RPM to extinguish ALT 2 caution light. ALT 2 caution light shall go out below RPM.
Voltage\dotfill CHECK\\
Magnetos\dotfill CHECK Left and Right\\
\hspace*{6mm} Ignition Switch\dotfill R, note RPM, then BOTH\\
\hspace*{6mm} Ignition Switch\dotfill L, note RPM, then BOTH\\
Power Lever\dotfill DECREASE to 1000 RPM\\
Transponder\dotfill ALT\\
Navigation Radios/GPS\dotfill SET for Takeoff\\
Pitot Heat\dotfill AS REQUIRED
\section*{NORMAL TAKEOFF}
Power Lever\dotfill FULL FORWARD\\
Engine Instruments\dotfill CHECK\\
Brakes\dotfill RELEASE (Steer with Rudder Only)\\
Elevator Control\dotfill ROTATE Smoothly at 70-73 KIAS\\
At 80 KIAS, Flaps\dotfill UP
\section*{SHORT FIELD TAKEOFF}
Flaps\dotfill 50\%\\
Brakes\dotfill HOLD\\
Power Lever\dotfill FULL FORWARD\\
Mixture\dotfill SET\\
Engine Instruments\dotfill CHECK\\
Brakes\dotfill RELEASE (Steer with Rudder Only)\\
Elevator Control\dotfill ROTATE Smoothly at 70 KIAS\\
Airspeed at Obstacle\dotfill 78 KIAS
\section*{CLIMB}
Climb Power\dotfill SET\\
Mixture\dotfill LEAN as required for altitude\\
Engine Instruments\dotfill CHECK\\
Fuel Pump\dotfill OFF
\section*{CRUISE}
Cruise Power\dotfill SET\\
Engine Instruments\dotfill MONITOR\\
Fuel Flow and Balance\dotfill MONITOR\\
Mixture\dotfill LEAN as required
\section*{DESCENT}
Altimeter\dotfill SET\\
Cabin Heat/Defrost\dotfill AS REQUIRED\\
Fuel System\dotfill CHECK\\
Mixture\dotfill AS REQUIRED\\
Flaps\dotfill AS REQUIRED\\
Brake Pressure\dotfill CHECK
\section*{BEFORE LANDING}
Mixture\dotfill FULL RICH\\
Fuel Pump\dotfill BOOST\\
Flaps\dotfill AS REQUIRED\\
Landing Light\dotfill AS REQUIRED\\
Autopilot\dotfill DISENGAGE
\section*{NORMAL LANDING}
Normal landings are made with full flaps with power on or off. Surface winds and air turbulence are usually the primary factors in determining the most comfortable approach speed.\\
Actual touchdown should be made with power off and on the main wheels first to reduce the landing speed and subsequent need for braking. Gently lower the nose wheel to the runway after airplane speed has dimished. This is espcially important for rough or soft field landings.
\section*{SHORT FIELD LANDING}
For a short field landing in smooth air conditions, make an approach at 77 KIAS with full flaps using enough power to control the glide path (slightly higher approach speeds should be used under turbulent air conditions). After all approach obstacles are cleared, progressively reduce power to reach idle just before touchdown and maintain the approach speed by lowering the nose of the airplane. Touchdown should be made power-off and on the main wheels first. Immediately after touchdown, lower the nose wheel and apply braking as required. For maximum brake effectiveness, retract the flaps, hold the control yoke full back and apply maximum brake pressure without skidding.
\section*{CROSSWIND LANDING}
Normal crosswind landings are made with full flaps. Avoid prolonged slips. After touchdown, hold a straight course with rudder and brakes as required. The maximum allowable crosswind velocity is dependent upon pilot capability as well as aircraft limitations. Operation in direct crosswind of 20 knows has been demonstrated.
\section*{BALKED LANDING}
Autopilot\dotfill DISENGAGE\\
Power Lever\dotfill FULL FORWARD\\
Flaps\dotfill 50\%\\
Airspeed\dotfill 75-80 KIAS\\
Flaps\dotfill UP (After clear of obstacles)
\section*{AFTER LANDING}
Flaps\dotfill UP\\
Power Lever\dotfill 1000 RPM\\
Transponder\dotfill STBY\\
Pitot Heat\dotfill OFF\\
Fuel Pump\dotfill OFF\\
\section*{SHUTDOWN}
Avionics Switch\dotfill OFF\\
Fuel Pump (if used)\dotfill OFF\\
Mixture\dotfill CUTOFF\\
Magnetos\dotfill OFF\\
Bat-Alt Master Switches\dotfill OFF\\
ELT\dotfill TRANSMIT LIGHT OUT
\vfill\null
\vspace*{-1em}
\section*{MAXIMUM POWER FUEL FLOW}
\begin{center}
\begin{tabular}{|c|c|c|c|c|c|}
    \hline
    \rowcolor[HTML]{9B9B9B} 
    {\color[HTML]{333333} \begin{tabular}[c]{@{}c@{}}Pressure\\ Altitude\end{tabular}} & {\color[HTML]{333333} \begin{tabular}[c]{@{}c@{}}Target\\ Fuel\\ Flow\end{tabular}} & {\color[HTML]{333333} \begin{tabular}[c]{@{}c@{}}Pressure\\ Altitude\end{tabular}} & {\color[HTML]{333333} \begin{tabular}[c]{@{}c@{}}Target\\ Fuel\\ Flow\end{tabular}} & {\color[HTML]{333333} \begin{tabular}[c]{@{}c@{}}Pressure\\ Altitude\end{tabular}} & {\color[HTML]{333333} \begin{tabular}[c]{@{}c@{}}Target\\ Fuel \\ Flow\end{tabular}} \\ \hline
    0 & 27.1 & 7000 & 21.4 & 14,000 & 17.5 \\ \hline
    1000 & 26.2 & 8000 & 20.5 & 15,000 & 16.9 \\ \hline
    2000 & 25.1 & 9000 & 19.9 & 16,000 & 16.7 \\ \hline
    3000 & 24.3 & 10,000 & 19.5 & 17,000 & 16.2 \\ \hline
    4000 & 23.6 & 11,000 & 18.8 & 17,500 & 16.1 \\ \hline
    5000 & 22.8 & 12,000 & 18.4 & \cellcolor[HTML]{9B9B9B} & \cellcolor[HTML]{9B9B9B} \\ \hline
    6000 & 22.1 & 13,000 & 17.9 & \cellcolor[HTML]{9B9B9B} & \cellcolor[HTML]{9B9B9B} \\ \hline
    \end{tabular}
\end{center}
\section*{AIRSPEED CALCULATION - NORMAL STATIC SOURCE}
\begin{center}
    \begin{tabular}{|l|l|l|l|}
        \hline
        \rowcolor[HTML]{9B9B9B} 
        \multicolumn{1}{|c|}{\cellcolor[HTML]{9B9B9B}{\color[HTML]{333333} }} & \multicolumn{3}{c|}{\cellcolor[HTML]{9B9B9B}{\color[HTML]{333333} KCAS}} \\ \cline{2-4} 
        \rowcolor[HTML]{9B9B9B} 
        \multicolumn{1}{|c|}{\multirow{-2}{*}{\cellcolor[HTML]{9B9B9B}{\color[HTML]{333333} KIAS}}} & \multicolumn{1}{c|}{\cellcolor[HTML]{9B9B9B}\begin{tabular}[c]{@{}c@{}}Flaps\\ 10\%\end{tabular}} & \multicolumn{1}{c|}{\cellcolor[HTML]{9B9B9B}\begin{tabular}[c]{@{}c@{}}Flaps\\ 50\%\end{tabular}} & \multicolumn{1}{c|}{\cellcolor[HTML]{9B9B9B}\begin{tabular}[c]{@{}c@{}}Flaps\\ 100\%\end{tabular}} \\ \hline
        \multicolumn{1}{|c|}{60} & \multicolumn{1}{c|}{\cellcolor[HTML]{9B9B9B}{\color[HTML]{9B9B9B} }} & \multicolumn{1}{c|}{\cellcolor[HTML]{9B9B9B}} & \multicolumn{1}{c|}{58} \\ \hline
        \multicolumn{1}{|c|}{70} & \multicolumn{1}{c|}{\cellcolor[HTML]{9B9B9B}{\color[HTML]{9B9B9B} }} & \multicolumn{1}{c|}{68} & \multicolumn{1}{c|}{69} \\ \hline
        \multicolumn{1}{|c|}{80} & \multicolumn{1}{c|}{79} & \multicolumn{1}{c|}{80} & \multicolumn{1}{c|}{80} \\ \hline
        \multicolumn{1}{|c|}{90} & \multicolumn{1}{c|}{90} & \multicolumn{1}{c|}{91} & \multicolumn{1}{c|}{90} \\ \hline
        \multicolumn{1}{|c|}{100} & \multicolumn{1}{c|}{100} & \multicolumn{1}{c|}{101} & \multicolumn{1}{c|}{100} \\ \hline
        \multicolumn{1}{|c|}{110} & \multicolumn{1}{c|}{110} & \multicolumn{1}{c|}{111} & \multicolumn{1}{c|}{\cellcolor[HTML]{9B9B9B}} \\ \hline
        120 & 121 & 121 & \cellcolor[HTML]{9B9B9B} \\ \hline
        130 & 131 & \cellcolor[HTML]{9B9B9B} & \cellcolor[HTML]{9B9B9B} \\ \hline
        140 & 142 & \cellcolor[HTML]{9B9B9B} & \cellcolor[HTML]{9B9B9B} \\ \hline
        150 & 152 & \cellcolor[HTML]{9B9B9B} & \cellcolor[HTML]{9B9B9B} \\ \hline
        160 & 162 & \cellcolor[HTML]{9B9B9B} & \cellcolor[HTML]{9B9B9B} \\ \hline
        170 & 172 & \cellcolor[HTML]{9B9B9B} & \cellcolor[HTML]{9B9B9B} \\ \hline
        180 & 183 & \cellcolor[HTML]{9B9B9B} & \cellcolor[HTML]{9B9B9B} \\ \hline
        190 & 193 & \cellcolor[HTML]{9B9B9B} & \cellcolor[HTML]{9B9B9B} \\ \hline
        200 & 203 & \cellcolor[HTML]{9B9B9B} & \cellcolor[HTML]{9B9B9B} \\ \hline
    \end{tabular}
\end{center}
\end{multicols*}
\newpage
\begin{center}
    {\fontsize{20.74}{70}\selectfont Emergency Procedures Checklist}
    \break
    \color{Red}
    {\fontsize{10}{70}\selectfont For simulation use only, not for real world flight}
\end{center}
\vspace*{-2em}
\begin{multicols*}{2}
\section*{AIRSPEEDS FOR EMERGENCY OPERATIONS}
Maneuvering Speed:\\
\hspace*{6mm} 3400 lb \dotfill 133 KIAS\\
Best Glide:\\
\hspace*{6mm} 3400 lb \dotfill 88 KIAS\\
\hspace*{6mm} 2900 lb \dotfill 87 KIAS\\
Emergency Landing (Engine-Out):\\
\hspace*{6mm} Flaps Up \dotfill 90 KIAS\\
\hspace*{6mm} Flaps 50\% \dotfill 85 KIAS\\
\hspace*{6mm} Flaps 100\% \dotfill 80 KIAS
\section*{ENGINE FIRE DURING START}
Mixture\dotfill CUTOFF\\
Fuel Pump\dotfill OFF\\
Fuel Selector\dotfill OFF\\
Power Lever\dotfill FORWARD\\
Starter\dotfill CRANK\\
If flames persist, perform Emergency Engine Shutdown on Ground and Emergency Ground Egress checklists
\section*{BRAKE FAILURE DURING TAXI}
Engine Power\dotfill AS REQUIRED\\
\hspace{6mm} To stop airplane\dotfill REDUCE\\
\hspace{6mm} If necessary for steering\dotfill INCREASE\\
Directional Control\dotfill MAINTAIN WITH RUDDER\\
Brake Pedal(s)\dotfill PUMP
\section*{ABORTED TAKEOFF}
Power Lever\dotfill IDLE\\
Brakes\dotfill AS REQUIRED
\section*{EMERGENCY ENGINE SHUTDOWN ON GROUND}
Power Lever\dotfill IDLE\\
Fuel Pump (if used)\dotfill OFF\\
Mixture\dotfill CUTOFF\\
Fuel Selector\dotfill OFF\\
Ignition Switch\dotfill OFF\\
Bat-Alt Master Switches\dotfill OFF
\section*{ENGINE FAILURE ON TAKEOFF (LOW ALTITUDE)}
Best Glide or Landing Speed (as apprpriate)\dotfill ESTABLISH\\
Mixture\dotfill CUTOFF\\
Fuel Selector\dotfill OFF\\
Ignition Switch\dotfill OFF\\
Flaps\dotfill AS REQUIRED\\
If time permits:\\
Power Lever\dotfill IDLE\\
Fuel Pump\dotfill OFF\\
Bat-Alt Master Switches\dotfill OFF
\section*{ENGINE FAILURE IN FLIGHT}
Best Glide Speed\dotfill ESTABLISH\\
Mixture\dotfill FULL RICH\\
Fuel Selector\dotfill SWITCH TANKS\\
Fuel Pump\dotfill BOOST\\
Alternate Induction Air\dotfill ON\\
Ignition Switch\dotfill CHECK, BOTH\\
If engine does not start, proceed to Engine Airstart or Forced Landing checklist, as required
\vfill\null
\section*{ENGINE AIRSTART}
Bat Master Switches\dotfill ON\\
Power Lever\dotfill OPEN 1/2 INCH\\
Mixture\dotfill RICH\\
Fuel Selector\dotfill SWITCH TANKS\\
Ignition Switch\dotfill BOTH\\
Fuel Pump\dotfill BOOST\\
Alt Master Switches\dotfill OFF\\
Starter (Propeller not Windmilling)\dotfill ENGAGE\\
Power Lever\dotfill slowly INCREASE\\
Alt Master Switches\dotfill ON\\
If engine will not start, perform Forced Landing checklist
\section*{ENGINE PARTIAL POWER LOSS}
Fuel Pump\dotfill BOOST\\
Fuel Selector\dotfill SWITCH TANKS\\
Mixture\dotfill CHECK appropriate for flight conditions\\
Power Lever\dotfill SWEEP\\
Alternate Induction Air\dotfill ON\\
Ignition Switch\dotfill BOTH, L, then R\\
Land as soon as practical
\section*{LOW OIL PRESSURE}
Power Lever\dotfill MINIMUM REQUIRED\\
Land as soon as practical
\section*{PROPELLER GOVERNOR FAILURE}
Propeller RPM will not increase:\\
\hspace*{6mm} Oil Pressure\dotfill CHECK\\
\hspace*{6mm} Land as soon as practical\\
Propeller overspeeds or will not decrease:\\
\hspace*{6mm} Power Lever\dotfill ADJUST (to keep RPM in limits)\\
\hspace*{6mm} Airspeed\dotfill REDUCE to 90 KIAS\\
\hspace*{6mm} Land as soon as practical
\section*{SMOKE AND FUME ELIMINATION}
Heater\dotfill OFF\\
Air Vents\dotfill OPEN, FULL COLD\\
Prepare to land as soon as possible
\section*{ENGINE FIRE IN FLIGHT}
Mixture \dotfill CUTOFF\\
Fuel Pump\dotfill OFF\\
Power Lever\dotfill IDLE\\
Fuel Selector\dotfill OFF\\
Ignition Switch\dotfill OFF\\
Perform Forced Landing checklist
\section*{WING FIRE IN FLIGHT}
Pitot Heat Switch\dotfill OFF\\
Navigation Light Switch\dotfill OFF\\
Strobe Light Switch\dotfill OFF\\
If possible, side slip to keep flames away from fuel tank and cabin\\
\textbf{Note:} Outting the airplane into a dive may blow out the fire. Do not exceed V\textsubscript{NE} during the dive\\
Land as soon as possible
\vfill\null
\section*{CABIN FIRE IN FLIGHT}
Bat-Alt Master Switches\dotfill OFF, AS REQ'D\\
Heater\dotfill OFF\\
Air Vents\dotfill CLOSED\\
Fire Extinquisher\dotfill ACTIVATE\\
When fire extinguished, Air Vents\dotfill OPEN, FULL COLD\\
Avionics Power Switch\dotfill OFF\\
All other switches\dotfill OFF\\
Land as soon as possible\\
If setting master switches off elimiated source of fire or fumes and airplane is in night, weather, or IFR condtions:\\
\hspace*{6mm} Bat-Alt Master Switches\dotfill ON\\
\hspace*{6mm} Avionics Power Switch\dotfill ON\\
\hspace*{6mm} Activate required systems one at a time. Pause several seconds between activating each system to isolate malfunctioning system.
\section*{INADVERTENT ICING ENCOUNTER}
Pitot Heat\dotfill ON\\
Exit icing conditions. Turn back or change altitude\\
Cabin Heat\dotfill MAXIMUM\\
Windshield Defrost\dotfill FULL OPEN\\
Alternate Induction Air\dotfill ON
\section*{EMERGENCY DESCENT}
Power Lever\dotfill IDLE\\
Mixture\dotfill As Required\\
Airspeed\dotfill V\textsubscript{NE} (201 KIAS)
\section*{INADVERTENT IMC ENCOUNTER}
Airplane control\dotfill Establish Straight and Level FLight\\
Autopilot\dotfill Engage to hold Heading and Altitude\\
Heading\dotfill Reset to initiate 180° turn
\section*{INADVERTENT SPIRAL DIVE DURING IMC FLIGHT}
Power Lever\dotfill IDLE\\
Stop the spiral dive by using coordinated aileron and rudder control while referring to the attitude indicator and turn coordinator to level the wings\\
Cautiosuly apply elevator back perssure to bring airplace to level flight attitude\\
Trim for level flight\\
Set power as required\\
Use autopilot if functional otherwise keep hands off control yoke, use rudder to hold constant heading\\
Exit IMC conditions as soon as possible
\section*{FORCED LANDING}
Best Glide Speed\dotfill ESTABLISH\\
Radio\dotfill Transmit (121.5 MHz) MAYDAY giving location and intentions\\
Transponder\dotfill SQUAWK 7700\\
If off airport, ELT\dotfill ACTIVATE\\
Power Lever\dotfill IDLE\\
Mixture\dotfill CUTOFF\\
Fuel Selector\dotfill OFF\\
Ignition Switch\dotfill OFF\\
Fuel Pump\dotfill OFF\\
Flaps (when landing is assured)\dotfill 100\%\\
Master Switches\dotfill OFF
\section*{LANDING WITHOUT ELEVATOR CONTROL}
Flaps\dotfill SET 50\%\\
Trim\dotfill SET 80 KIAS\\
Power\dotfill AS REQUIRED FOR GLIDE ANGLE
\vfill\null
\section*{LANDING WITH FAILED BRAKES}
\textbf{One brake inoperative}\\
Land on the side of the runway corresponding to the inoperative brake\\
Maintain directional control using rudder and working brake\\
\textbf{Both brakes inoperative}\\
Divert to the longest, widest runway with the most direct headwind\\
Land on downwind side of the runway\\
Use the rudder for obstacle avoidance\\
Perform Emergency Engine Shutdown on Ground checklist
\section*{LANDING WITH FLAT TIRE}
\textbf{Main Gear}\\
Land on the side of the runway corresponding to the good tire\\
Maintain directional control with the brakes and rudder\\
Do not taxi. Stop the airplane and perform a normal engine shutdown
\textbf{Nose Gear}\\
Land in the center of the runway\\
Hold the nosewheel off the ground as long as possible\\
Do not taxi. Stop the airplane and perform a normal engine shutdown
\section*{ALT 1 LIGHT STEADY}
ALT 1 Master Switch\dotfill OFF\\
Alternator 1 Circuit Breaker\dotfill CHECK and RESET\\
ALT 1 Master Switch\dotfill ON\\
If alternator does not reset:\\
Switch off unnecessary equipment on Main Bus 1, Main Bus 2, and the Non-Essential Buses to reduce load. Monitor voltage\\
ALT 1 Master Switch\dotfill OFF\\
Land as soon as practical
\section*{ALT 1 LIGHT FLASHING}
Ammeter Switch\dotfill BATT\\
If charging rate is greater than 30 amps, reduce load on Main Bus 1, Main Bus 2, and Non-Essential buses\\
Monitor ammerter until battery charge rate is less than 15 amps\\
When battery charge rate is within limits, add loads as necessary for flight conditions
\section*{ALT 2 LIGHT STEADY}
ALT 2 Master Switch\dotfill OFF\\
Alternator 2 Circuit Breaker\dotfill CHECK and RESET\\
ALT 2 Master Switch\dotfill ON\\
If alternator does not reset:\\
Switch off unnecessary equipment on Main Bus 1, Main Bus 2, and the Non-Essential Buses to reduce load. Monitor voltage\\
ALT 2 Master Switch\dotfill OFF\\
Land as soon as practical
\section*{PITOT STATIC MALFUNCTION}
Pitot Heat\dotfill ON\\
Alternate Static Source\dotfill OPEN
\section*{ELECTRIC TRIM/AUTOPILOT FAILURE}
Airplane Control\dotfill MAINTAIN MANUALLY\\
Autopilot (if engaged)\dotfill Disengage\\
If Problem Is Not Corrected:
Circuit Breakers\dotfill PULL AS Required\\
\hspace*{6mm} PITCH TRIM\\
\hspace*{6mm} ROLL TRIM\\
\hspace*{6mm} AUTOPILOT\\
Power Lever\dotfill AS REQUIRED\\
Control Yoke \dotfill MANUALLY HOLD PRESSURE\\
Land as soon as practical
\end{multicols*}
\end{document}